% Análise Preditiva como Base para uma Abordagem Proativa de 
% Elasticidade em Cloud Computing
% PAULO IGOR LEOPOLDES BENDER
%
% Monografia - Trabalho de Conclusão de Curso I
% Bacharelado em Ciência da Computação
% 
% Elaborado com base nas orientações dadas no documento
% ``GUIA PARA ELABORAÇÃO DE TRABALHOS ACADÊMICOS''
% disponível no site da biblioteca da Unisinos.
% http://www.unisinos.br/biblioteca
%

%=======================================================================
% Declarações iniciais identificando a classe de documento e
% selecionando alguns pacotes adicionais.
%
% As opções disponíveis (separe-as com vírgulas, sem espaço) são:
% - twoside: Formata o documento para impressão frente-e-verso
%   (o default é somente-frente)
% - english,brazilian,french,german,etc.: idiomas usados no documento.
%   Deve ser colocado por último o idioma principal.
%=======================================================================
\documentclass[twoside,english,brazilian]{UNISINOSmonografia}
\usepackage[utf8]{inputenc} % charset do texto (utf8, latin1, etc.)
\usepackage[T1]{fontenc} % encoding da fonte (afeta a sep. de sílabas)
\usepackage{graphicx} % comandos para gráficos e inclusão de figuras
\usepackage{bibentry} % para inserir refs. bib. no meio do texto

%=======================================================================
% Escolha do sistema para geração de referências bibliográficas.
%
% O default é usar o estilo unisinos.bst.  Comente a definição abaixo
% e descomente a linha seguinte para usar o estilo do ABNTeX (é
% necessário ter esse pacote instalado).
%
% A vantagem do unisinos.bst é que ele permite o uso de um arquivo .bib
% seguindo as orientações tradicionais do BibTeX (veja essas orientações
% em http://ctan.tug.org/tex-archive/biblio/bibtex/contrib/doc/btxdoc.pdf).
% Entretanto, o estilo não suporta algumas citações mais exóticas como
% apud.  Para isso, use o ABNTeX, mas esteja ciente de que muitas de
% suas referências serão incompatíveis com os estilos tradicionais do
% BibTeX como plain, alpha, ieeetr, entre outros.
%=======================================================================
\unisinosbst
%\usepackage[alf]{abntcite}

%=======================================================================
% Dados gerais sobre o trabalho.
%=======================================================================
\autor{Bender}{Paulo Igor Leopoldes}
\titulo{Análise Preditiva como Base para uma Abordagem Proativa de Elasticidade em Cloud Computing}
\subtitulo{Versão \LaTeX}
\orientador[Prof.~Dr.]{Righi}{Rodrigo da Rosa}
%\coorientador[Prof.~Dr.]{Lamport}{Leslie}
\local{São Leopoldo}
\ano{2014}

%% dados específicos para Dissertação de Mestrado
%\unidade{Unidade Acadêmica de Pesquisa e Pós-Graduação}
%\curso{Programa de Pós-Graduação em Computação Aplicada}
%\nivel{Nível Mestrado}
%\natureza{%
%Dissertação apresentada como requisito parcial para a obtenção
%do título de Mestre pelo Programa de Pós-Graduação em Computação
%Aplicada da Universidade do Vale do Rio dos Sinos --- UNISINOS
%}
%% dados da ficha catalográfica (obrigatória somente para diss. e tese)
%\cip{Dissertação (mestrado)}{004.732}
%\bibliotecario{Bibliotecária responsável: Fulana da Silva}{12/3456}

%% dados específicos para monografia de Graduação
\unidade{Unidade Acadêmica Graduação}
\curso{Curso de Bacharelado em Ciência da Computação}
\natureza{%
Trabalho de Conclusão de Curso apresentado como requisito parcial
para a obtenção do título de Bacharel em Ciência da Computação
pela Universidade do Vale do Rio dos Sinos --- UNISINOS
}

% cada palavra-chave deve ser fornecida duas vezes, uma em português e
% outra no idioma estrangeiro (na verdade, em tantos idiomas quantos se
% desejar).
\palavrachave{brazilian}{Computação em Nuvem}
\palavrachave{brazilian}{Elasticidade}
\palavrachave{brazilian}{Análise Preditiva}
\palavrachave{brazilian}{Gerência de Recursos}
\palavrachave{english}{Cloud Computing}
\palavrachave{english}{Elasticity}
\palavrachave{english}{Predictive Analysis}
\palavrachave{english}{Resource provisioning}

%=======================================================================
% Início do documento.
%=======================================================================
\begin{document}
\capa
\folhaderosto
%\folhadeaprovacao % não deve ser incluída nos TCCs

%=======================================================================
% Dedicatória (opcional).
%
% O texto é normalmente colocado na parte de baixo da página, alinhado
% à direita.  Mas a formatação é basicamente livre.  Só não se escreve
% a palavra 'dedicatória'.
%=======================================================================
%\begin{dedicatoria}
%Aos nossos pais.\\[4ex] % quebra a linha dando um espaçamento maior
%\begin{itshape} % faz o texto ficar em itálico
%If I have seen farther than others,\\
%it is because I stood on the shoulders of giants.\\
%\end{itshape}
%--- \textsc{Sir Isaac Newton} % \textsc é o "small caps"
%\end{dedicatoria}

%=======================================================================
% Agradecimentos (opcional).
%=======================================================================
%\begin{agradecimentos}
%Obrigado!
%\end{agradecimentos}

%=======================================================================
% Epígrafe (opcional).
%
% ``[...] o autor apresenta uma citação, seguida de indicação de autoria,
% relacionada com a matéria tratada no corpo do trabalho. Podem, também,
% constar epígrafes nas folhas de aberturas das seções primárias.''
%=======================================================================
%\begin{epigrafe}
%``\textit{Ninguém abre um livro sem que aprenda alguma coisa}''.\\
%(Anônimo)
%\end{epigrafe}

%=======================================================================
% Resumo em Português.
%
% A recomendação é para 150 a 500 palavras.
%=======================================================================
\begin{abstract}
A própria definição de elasticidade se confunde com a definição de computação em nuvem. Sendo provavelmente, se não a característica mais significativa, uma das mais importantes. Este aspecto da cloud computing consiste na capacidade de aumentar ou diminuir a alocação de recursos, sejam eles instâncias de máquinas virtuais ou configurações de CPU, memória, larguara de banda de rede, etc. E tais alterações devem se dar, preferencialmente, em tempo de execução e sem interrupções com o objetivo de atender requisitos de demanda e gerenciar os recursos de forma inteligente. A elasticidade e a escalabilidade levam ao usuário uma sensação de que seus recursos são ilimitados, enquanto que cabe ao provedor destes serviços, garantir sob condições diversas e dinâmicas que esta sensação seja mantida, enquanto busca minimizar os seus próprios custos operacionais. Logo, a utilização mais eficiente dos recursos computacionais é do interesse de ambos, e nisto a elasticidade tem um papel fundamental. Um dos grandes desafios que esta área apresenta é justamente a aplicação do princípio da elasticidade de maneira automática e proativa. Dada a complexidade de compreender quais requisitos são demandados por uma aplicação e seus diferentes estados ao longo do tempo. Os maiores provedores de computação em nuvem, em seus diferentes formatos, oferecem além do controle de escalabilidade manual, um controle automático mas de maneira reativa, o qual embora eficaz, não é tão eficiente quanto um controle proativo que tenha ciência das necessidades de recursos para tratar com certa previsibilidade a demanda futura de uma dada aplicação. O objetivo desta monografia é, a partir de um estudo das pesquisas recentes no campo da elasticidade proativa, juntamente com uma fundamentação em modelos matemáticos, propor uma abordagem de análise preditiva a partir dos dados de utilização dos serviços de cloud computing com o intuito de fornecer subsídios para a implementação de uma abordagem proativa para a questão da elasticidade.
\end{abstract}

%=======================================================================
% Resumo em língua estrangeira (obrigatório somente para teses e
% dissertações).
%
% O idioma usado aqui deve necessariamente aparecer nos parâmetros do
% \documentclass, no início do documento.
%=======================================================================
%\begin{otherlanguage}{english}
%\begin{abstract}
%This document presents guidelines on the use of UNISINOS's \LaTeX\ class for academic reports and dissertations.  At the same time, it %serves as an example on using the class, employing the main commands and providing further general orientations on the use of \LaTeX.  %In addition, we have added guidelines for the process of writing itself, collecting tips and recommendations that contribute to the %technical quality enhancement of academic monographs.  The Abstract should be composed of 150 to 500~words and must not contain any %citations.  It is suggested that a single paragraph be used.
%\end{abstract}
%\end{otherlanguage}

%=======================================================================
% Lista de Figuras (opcional).
%=======================================================================
%\listoffigures

%=======================================================================
% Lista de Tabelas (opcional).
%=======================================================================
%\listoftables

%=======================================================================
% Lista de Abreviaturas (opcional).
%
% Deve ser passada como parâmetro a maior das abreviaturas utilizadas.
%=======================================================================
%\begin{listadeabreviaturas}{seg., segs.}
%\item[ampl.] ampliado, -a
%\item[atual.] atualizado, -a
%\item[coord.] coordenador
%\item[N.~T.] Novo Testamento
%\item[seg., segs.] seguinte, -s
%\end{listadeabreviaturas}

%=======================================================================
% Lista de Siglas (opcional).
%
% Deve ser passada como parâmetro a maior das siglas utilizadas.
%=======================================================================
%\begin{listadesiglas}{FAPERGS}
%\item[ABNT] Associação Brasileira de Normas Técnicas
%\item[CAPES] Coordenação de Aperfeiçoamento de Pessoal de Nível Superior
%\item[FAPERGS] Fundação de Amparo à Pesquisa do Estado do Rio Grande do Sul
%\end{listadesiglas}

%=======================================================================
% Lista de Símbolos (opcional).
%
% Deve ser passado o maior (mais largo) dos símbolos utilizados.
%=======================================================================
%\begin{listadesimbolos}{Ca}
%\item[\textsuperscript{o}C] Graus Celsius
%\item[Al] Alumínio
%\item[Ca] Cálcio
%\end{listadesimbolos}

%=======================================================================
% Sumário
%=======================================================================
\tableofcontents

%=======================================================================
% Introdução
%=======================================================================
\chapter{Introdução}

% as epígrafes nos capítulos são opcionais
%\epigrafecap{The reasonable man adapts himself to the world; the unreasonable one persists in trying to adapt the world to himself. %Therefore all progress depends on the unreasonable man.}{George Bernard Shaw}

A computação em nuvem trouxe um novo modelo de relacionamento entre as empresas e os seus recursos computacionais. A gestão destes recursos sempre foi uma questão bastante delicada, que tem um reflexo direto no gerenciamento dos custos das companhias. E embora haja organizações que têm uma demanda por recursos computacionais com um comportamento mais previsível, é bastante significativo o número de empresas que têm necessidades mais dinâmicas neste aspecto.

Um exemplo clássico de comportamento com alto grau de variação é encontrado no comércio eletrônico. A sazonalidade com que os recursos são demandados faz com que as empresas gastem muito dinheiro em infraestrutura para garantir o desempenho desejado durante os picos de acesso, e fiquem com recursos parados durante a baixa temporada. É claro, que este cenário não é único do comércio eletrônico, na verdade a maioria das empresas, em maior ou menor grau, acabam apresentando um comportamento semelhante a este no que diz respeito às suas necessidades computacionais.

A solução oferecida pela cloud computing é justamente disponibilizar recursos computacionais para as empresas de acordo com suas respectivas demandas. Desta forma, os usuários não precisam arcar com o alto custo de manter um parque de máquinas, ou mesmo com as questões relacionadas a gestão e aquisição delas. E tudo isto a um custo bastante baixo, com uma disponibilidade imediata de contratação e um gerenciamento bastante simplificado.

Neste contexto, uma das características mais importantes da computação em nuvem é a elasticidade. Ou seja, a capacidade do ambiente computacional aumentar ou diminuir os recursos demandados pelo usuário. Esta capacidade se apresenta tanto em nível de instâncias de máquinas virtuais e/ou nós de computação, quanto em nível de CPU, memória e largura de banda de rede.Uma lista mais completa das características essenciais de um ambiente em nuvem pode ser encontrada em \cite{coutinho2013elasticidade}, que cita \textit{"elasticidade rápida"} como uma destas características.

A política de alocação de recursos em ambientes de computação em nuvem pode ser tanto manual quanto automática. Para a abordagem automática, temos ainda duas classificações: reativa e proativa. Na primeira, o usuário define marcos (thresholds) para um mecanismo de regra-condição-ação através dos quais o sistema aloca mais recursos (threshold superior) ou os consolida (threshold inferior). Embora, esta primeira abordagem seja chamada de automática, se faz necessária a intervenção do usuário para especificar as regras, bem como conhecimento da ferramenta de nuvem. Por outro lado, a abordagem proativa usa técnicas de predição para identificar padrões e antecipar a alocação ou consolidação de recursos de forma automática, sem a intervenção do usuário.

A abordagem reativa, juntamente com as políticas manuais, é mais comum em sistemas comerciais de nuvem disponíveis na web. Enquanto que existe um esforço de pesquisas na academia para propor soluções proativas para a questão da elasticidade. Portanto, um dos grandes desafios para os provedores é justamente retirar a necessidade de intervenção do usário e oferecer mecanismos que implementem a elasticidade proativa. O que traria um ajuste mais fino na economia de recursos, dentre outros benefícios.

\section{Questão de Pesquisa}
Um dos grandes pontos a serem discutidos, no que diz respeito à implementação de um tratamento automático para a elasticidade, é a questão da análise do comportamento da aplicação. A qual oferece seus próprios obstáculos a serem transpostos.
Uma técnica de auto-escalonamento que tenta antecipar os requisitos futuros de uma aplicação, baseado na análise de seu desempenho, é portanto, uma abordagem preditiva. Na literatura, este tema tem sido discutido sob diversos pontos de vista, como nos mostra o trabalho de \cite{Lorido-botr2012}. Algumas das técnicas sugeridas apresentam tanto uma natureza reativa quanto proativa, enquanto outas são excencialmente proativas.

A análise preditiva baseada em séries temporais é uma destas estratégias que têm uma abordagem puramente proativa. Uma série temporal é uma coleção de observações, as quais podem ser discretizadas ou não, feitas sequencialmente ao longo de um período de tempo. Ou como definido por \cite{shumway2000time}, séries temporais podem ser definidas como uma coleção de variáveis aleatórias indexadas de acordo com a ordem em que elas são obtidas no tempo. 

A análise de séries temporais pode ser usada para encontrar padrões de repetição na carga de trabalho de uma aplicação, ou para tentar prever valores futuros. 

%TODO: sei que falta alguma coisa aqui para complementar e deixar mais claro a questão da pesquisa

\section{Objetivos}
O objetivo deste trabalho é apresentar uma proposta para análise preditiva que possa servir de base na implementação da elasticidade proativa em ambientes de computação em nuvem.

Para atingir o objetivo acima citado, se faz necessário primeiramente: desenvolver uma estratégia para coleta de dados, criar uma estrutura para analisar os dados e aplicar modelos matemáticos condizentes com os padrões encontrados. E por último, disponibilizar como resposta uma sugestão de alteração da configuração vigente, em virtude da demandada prevista de acordo com a análise realizada. 

\section{Organização do Texto}
Este texto está organizado em cinco seções. Primeiramente, na seção 2, iremos revisar questões básicas a respeito da área da computação em nuvem, e as definições que são pertinentes ao escopo deste trabalho. Uma retomada dos tópicos relacionados às séries temporais também se faz necessária, bem como questões gerais a respeito de análise preditiva. Na seção 3, iremos avaliar trabalhos correlatos na área, e suas contribuições para a questão da elasticidade proativa. Esta avaliação tomará como objetivo traçar um quadro comparativo das soluções propostas, suas divergências e convergências, com o intuito de identificar lacunas e oportunidades de trabalho na literatura analisada. A seção 4 trará o modelo proposto, descrevendo sua arquitetura e metodologia para avaliação. E por fim, a seção 5 apresenta uma conclusão, especificando as contribuições esperadas, apontando trabalhos relacionados e um cronograma para o projeto.

%=======================================================================
% Escrevendo o Texto
%=======================================================================
\chapter{Fundamentação Teórica}

\section{Séries Temporais}
Como regra geral, use os comandos tradicionais do \LaTeX\ para formatar seu texto.  Neste documento procuramos demonstrar os comandos mais comumente utilizados em monografias acadêmicas.

Neste capítulo apresentamos alguns exemplos de como colocar figuras e tabelas no seu texto.

\section{Predição}


%=======================================================================
% Exemplos de Citações e Referências Bibliográficas
%=======================================================================
\chapter{Exemplos de Citações e Referências Bibliográficas}
\nobibliography* % para usar o \bibentry
Neste capítulo são apresentados exemplos de citações e referências bibliográficas.  Aqui é utilizado o pacote \texttt{bibentry}, que permite a inserção de referências no meio do texto (atenção para a diferença entre citações e referências).

Você vai ver que, neste exemplo, não está sendo usado o estilo de referências bibliográficas do projeto ABNTeX\footnote{http://http://sourceforge.net/projects/abntex}.  Você é completamente livre para usá-lo (veja no início do arquivo .tex como fazer isso).  Os motivos para não usar o ABNTeX neste exemplo são basicamente dois:
\begin{itemize}
	\item Para usar o ABNTeX, é necessário instalá-lo em seu sistema \TeX\ primeiro; embora não seja uma tarefa tão complicada, enxergamos como uma dificuldade a mais para o usuário iniciante.  Nosso objetivo aqui é facilitar ao aluno da UNISINOS o uso deste modelo, de modo que basta copiar os arquivos \texttt{UNISINOSmonografia.cls} e \texttt{unisinos.bst} para a pasta onde estão seus arquivos .tex;
	\item As normas da ABNT são tão complexas que, para atender a todas as variações possíveis de citações e referências, o projeto ABNTeX criou uma série de campos adicionais nas entradas do arquivo .bib.  Embora funcione para o caso ABNT, o efeito colateral de fazer isso é que o seu arquivo .bib será muitas vezes incompatível com os demais estilos tradicionais do BibTeX, como \texttt{plain}, \texttt{alpha}, \texttt{ieeetr}, entre outros.  Por exemplo, em referências a artigos publicados em conferências, o campo \texttt{organization} é usado pelo ABNTeX para definir o nome do evento.  Isso não é padrão e não será reconhecido pelos estilos tradicionais\footnote{Veja como criar seus arquivos .bib no manual do BibTeX, que pode ser encontrado em http://ctan.tug.org/tex-archive/biblio/bibtex/contrib/doc/btxdoc.pdf.}.  Considerando que um dos maiores benefícios do BibTeX é criar um arquivo .bib que pode ser reutilizado pelo resto da vida, nossa estratégia com o \texttt{unisinos.bst} foi tentar aproximar ao máximo a formatação exigida pela ABNT sem implicar na criação de arquivos .bib incompatíveis.  Isso funciona bem na grande maioria dos casos, mas não em todos.  Nesse caso, a saída é usar o ABNTeX ou então alterar manualmente o arquivo .bbl que é gerado ao rodar o comando \texttt{bibtex}.
\end{itemize}

Em caso de dúvida, siga as orientações do manual da Biblioteca \cite{Biblioteca11} e, se necessário, da norma NBR~6023 \cite{NBR6023:2002}.

\section{Citações}
As citações podem ocorrer de duas formas: com os nomes dos autores inseridos no texto ou não.  Isso implica em uma construção diferente para as frases.  Por exemplo:
\begin{itemize}
	\item Com o nome do autor inserido no texto: ``De acordo com \citetexto{Tanenbaum03}, o modelo de referência OSI foi proposto de forma tardia.''
	\item Sem inserir o autor no texto: ``O modelo de referência OSI foi proposto de forma tardia \cite{Tanenbaum03}.''
\end{itemize}

\section{Livros}
Seguem alguns exemplos de referências de livros:
\begin{itemize}
	\item \bibentry{Buford09}.
	\item Livro com indicação de edição:\\
	\bibentry{Kurose10ptbr}.
\end{itemize}

\section{Artigos em Periódicos}
Os exemplos abaixo ilustram referências a artigos em periódicos.
\begin{itemize}
	\item \bibentry{Hayes08}.
	\item \bibentry{Lawton08}.
\end{itemize}

\section{Artigos em Conferências}
\begin{itemize}
	\item \bibentry{Laadan10}.
	\item \bibentry{Anderson95}.
\end{itemize}

\section{Teses e Dissertações}
Seguem algumas referências a trabalhos acadêmicos, como teses, dissertações, trabalhos de conclusão de curso, etc.
\begin{itemize}
	\item \bibentry{Teixeira09}.
	\item \bibentry{Flaumann05}.
\end{itemize}

%=======================================================================
% Referências
%=======================================================================
\bibliography{tcc1}


%=======================================================================
% Exemplo de Apêndice
% O Apêndice é utilizado para apresentar material complementar elaborado
% pelo próprio autor.  Deve seguir as mesmas regras de formatação do
% corpo principal do documento.
%=======================================================================
\appendix
\chapter{Informações Complementares}

O Apêndice é o lugar para incluir textos complementares, que não são essenciais para o entendimento do assunto principal da monografia, mas que podem contribuir com informação relevante (por exemplo, uma prova matemática, uma conceituação básica, etc.).  Ele deve seguir o formato normal do documento.

%=======================================================================
% Exemplo de Anexo
% O Anexo é utilizado para a ``inclusão de materiais não elaborados pelo
% próprio autor, como cópias de artigos, manuais, folders, balancetes, etc.
% e não precisam estar em conformidade com o modelo''.
%=======================================================================
\annex
\chapter{Artigos Publicados}
Existe diferença entre os Apêndices e os Anexos.  Os apêndices trazem informação escrita pelo próprio autor do trabalho, incorporando-se ao formato da monografia como um todo.  Já um anexo é um material à parte, definido/publicado por si só, e que o autor julga conveniente ser apresentado juntamente com a monografia.  Normalmente também vai apresentar formato próprio, como um artigo publicado, um folder, uma planilha, etc.
\end{document}
