% Análise Preditiva como Base para uma Abordagem Proativa de 
% Elasticidade em Cloud Computing
% PAULO IGOR LEOPOLDES BENDER
%
% Monografia - Trabalho de Conclusão de Curso I
% Bacharelado em Ciência da Computação
% 
% Elaborado com base nas orientações dadas no documento
% ``GUIA PARA ELABORAÇÃO DE TRABALHOS ACADÊMICOS''
% disponível no site da biblioteca da Unisinos.
% http://www.unisinos.br/biblioteca
%

%=======================================================================
% Declarações iniciais identificando a classe de documento e
% selecionando alguns pacotes adicionais.
%
% As opções disponíveis (separe-as com vírgulas, sem espaço) são:
% - twoside: Formata o documento para impressão frente-e-verso
%   (o default é somente-frente)
% - english,brazilian,french,german,etc.: idiomas usados no documento.
%   Deve ser colocado por último o idioma principal.
%=======================================================================
\documentclass[twoside,english,brazilian]{UNISINOSmonografia}
\usepackage[utf8]{inputenc} % charset do texto (utf8, latin1, etc.)
\usepackage[T1]{fontenc} % encoding da fonte (afeta a sep. de sílabas)
\usepackage{graphicx} % comandos para gráficos e inclusão de figuras
\usepackage{bibentry} % para inserir refs. bib. no meio do texto

%=======================================================================
% Escolha do sistema para geração de referências bibliográficas.
%
% O default é usar o estilo unisinos.bst.  Comente a definição abaixo
% e descomente a linha seguinte para usar o estilo do ABNTeX (é
% necessário ter esse pacote instalado).
%
% A vantagem do unisinos.bst é que ele permite o uso de um arquivo .bib
% seguindo as orientações tradicionais do BibTeX (veja essas orientações
% em http://ctan.tug.org/tex-archive/biblio/bibtex/contrib/doc/btxdoc.pdf).
% Entretanto, o estilo não suporta algumas citações mais exóticas como
% apud.  Para isso, use o ABNTeX, mas esteja ciente de que muitas de
% suas referências serão incompatíveis com os estilos tradicionais do
% BibTeX como plain, alpha, ieeetr, entre outros.
%=======================================================================
\unisinosbst
%\usepackage[alf]{abntcite}

%=======================================================================
% Dados gerais sobre o trabalho.
%=======================================================================
\autor{Bender}{Paulo Igor Leopoldes}
\titulo{Análise Preditiva como Base para uma Abordagem Proativa de Elasticidade em Cloud Computing}
%\subtitulo{}
\orientador[Prof.~Dr.]{Righi}{Rodrigo da Rosa}
%\coorientador[Prof.~Dr.]{Lamport}{Leslie}
\local{São Leopoldo}
\ano{2014}

%% dados específicos para Dissertação de Mestrado
%\unidade{Unidade Acadêmica de Pesquisa e Pós-Graduação}
%\curso{Programa de Pós-Graduação em Computação Aplicada}
%\nivel{Nível Mestrado}
%\natureza{%
%Dissertação apresentada como requisito parcial para a obtenção
%do título de Mestre pelo Programa de Pós-Graduação em Computação
%Aplicada da Universidade do Vale do Rio dos Sinos --- UNISINOS
%}
%% dados da ficha catalográfica (obrigatória somente para diss. e tese)
%\cip{Dissertação (mestrado)}{004.732}
%\bibliotecario{Bibliotecária responsável: Fulana da Silva}{12/3456}

%% dados específicos para monografia de Graduação
\unidade{Unidade Acadêmica Graduação}
\curso{Curso de Bacharelado em Ciência da Computação}
\natureza{%
Trabalho de Conclusão de Curso apresentado como requisito parcial
para a obtenção do título de Bacharel em Ciência da Computação
pela Universidade do Vale do Rio dos Sinos --- UNISINOS
}

% cada palavra-chave deve ser fornecida duas vezes, uma em português e
% outra no idioma estrangeiro (na verdade, em tantos idiomas quantos se
% desejar).
\palavrachave{brazilian}{Computação em Nuvem}
\palavrachave{brazilian}{Elasticidade}
\palavrachave{brazilian}{Análise Preditiva}
\palavrachave{brazilian}{Gerência de Recursos}
\palavrachave{english}{Cloud Computing}
\palavrachave{english}{Elasticity}
\palavrachave{english}{Predictive Analysis}
\palavrachave{english}{Resource provisioning}

%=======================================================================
% Início do documento.
%=======================================================================
\begin{document}
\capa
\folhaderosto
%\folhadeaprovacao % não deve ser incluída nos TCCs

%=======================================================================
% Dedicatória (opcional).
%
% O texto é normalmente colocado na parte de baixo da página, alinhado
% à direita.  Mas a formatação é basicamente livre.  Só não se escreve
% a palavra 'dedicatória'.
%=======================================================================
%\begin{dedicatoria}
%Aos nossos pais.\\[4ex] % quebra a linha dando um espaçamento maior
%\begin{itshape} % faz o texto ficar em itálico
%If I have seen farther than others,\\
%it is because I stood on the shoulders of giants.\\
%\end{itshape}
%--- \textsc{Sir Isaac Newton} % \textsc é o "small caps"
%\end{dedicatoria}

%=======================================================================
% Agradecimentos (opcional).
%=======================================================================
%\begin{agradecimentos}
%Obrigado!
%\end{agradecimentos}

%=======================================================================
% Epígrafe (opcional).
%
% ``[...] o autor apresenta uma citação, seguida de indicação de autoria,
% relacionada com a matéria tratada no corpo do trabalho. Podem, também,
% constar epígrafes nas folhas de aberturas das seções primárias.''
%=======================================================================
%\begin{epigrafe}
%``\textit{Ninguém abre um livro sem que aprenda alguma coisa}''.\\
%(Anônimo)
%\end{epigrafe}

%=======================================================================
% Resumo em Português.
%
% A recomendação é para 150 a 500 palavras.
%=======================================================================
\begin{abstract}
A própria definição de elasticidade se confunde com a definição de computação em nuvem. Sendo provavelmente, se não a característica mais significativa, uma das mais importantes. Este aspecto da cloud computing consiste na capacidade de aumentar ou diminuir a alocação de recursos, sejam eles instâncias de máquinas virtuais ou configurações de CPU, memória, largura de banda de rede, etc. E tais alterações devem se dar, preferencialmente, em tempo de execução e sem interrupções com o objetivo de atender requisitos de demanda e gerenciar os recursos de forma inteligente. A elasticidade e a escalabilidade levam ao usuário uma sensação de que seus recursos são ilimitados, enquanto que cabe ao provedor destes serviços, garantir sob condições diversas e dinâmicas que esta sensação seja mantida, enquanto busca minimizar os seus próprios custos operacionais. Logo, a utilização mais eficiente dos recursos computacionais é do interesse de ambos, e nisto a elasticidade tem um papel fundamental. Um dos grandes desafios que esta área apresenta é justamente a aplicação do princípio da elasticidade de maneira automática e proativa. Dada a complexidade de compreender quais requisitos são demandados por uma aplicação e seus diferentes estados ao longo do tempo. Os maiores provedores de computação em nuvem, em seus diferentes formatos, oferecem além do controle de escalabilidade manual, um controle automático mas de maneira reativa, o qual embora eficaz, não é tão eficiente quanto um controle proativo que tenha ciência das necessidades de recursos para tratar com certa previsibilidade a demanda futura de uma dada aplicação. O objetivo desta monografia é, a partir de um estudo das pesquisas recentes no campo da elasticidade proativa, juntamente com uma fundamentação em modelos matemáticos, propor uma abordagem de análise preditiva a partir dos dados de utilização dos serviços de cloud computing com o intuito de fornecer subsídios para a implementação de uma abordagem proativa para a questão da elasticidade.
\end{abstract}

%=======================================================================
% Resumo em língua estrangeira (obrigatório somente para teses e
% dissertações).
%
% O idioma usado aqui deve necessariamente aparecer nos parâmetros do
% \documentclass, no início do documento.
%=======================================================================
%\begin{otherlanguage}{english}
%\begin{abstract}
%This document presents guidelines on the use of UNISINOS's \LaTeX\ class for academic reports and dissertations.  At the same time, it %serves as an example on using the class, employing the main commands and providing further general orientations on the use of \LaTeX.  %In addition, we have added guidelines for the process of writing itself, collecting tips and recommendations that contribute to the %technical quality enhancement of academic monographs.  The Abstract should be composed of 150 to 500~words and must not contain any %citations.  It is suggested that a single paragraph be used.
%\end{abstract}
%\end{otherlanguage}

%=======================================================================
% Lista de Figuras (opcional).
%=======================================================================
%\listoffigures

%=======================================================================
% Lista de Tabelas (opcional).
%=======================================================================
%\listoftables

%=======================================================================
% Lista de Abreviaturas (opcional).
%
% Deve ser passada como parâmetro a maior das abreviaturas utilizadas.
%=======================================================================
%\begin{listadeabreviaturas}{seg., segs.}
%\item[ampl.] ampliado, -a
%\item[atual.] atualizado, -a
%\item[coord.] coordenador
%\item[N.~T.] Novo Testamento
%\item[seg., segs.] seguinte, -s
%\end{listadeabreviaturas}

%=======================================================================
% Lista de Siglas (opcional).
%
% Deve ser passada como parâmetro a maior das siglas utilizadas.
%=======================================================================
%\begin{listadesiglas}{FAPERGS}
%\item[ABNT] Associação Brasileira de Normas Técnicas
%\item[CAPES] Coordenação de Aperfeiçoamento de Pessoal de Nível Superior
%\item[FAPERGS] Fundação de Amparo à Pesquisa do Estado do Rio Grande do Sul
%\end{listadesiglas}

%=======================================================================
% Lista de Símbolos (opcional).
%
% Deve ser passado o maior (mais largo) dos símbolos utilizados.
%=======================================================================
%\begin{listadesimbolos}{Ca}
%\item[\textsuperscript{o}C] Graus Celsius
%\item[Al] Alumínio
%\item[Ca] Cálcio
%\end{listadesimbolos}

%=======================================================================
% Sumário
%=======================================================================
\tableofcontents

%=======================================================================
% Introdução
%=======================================================================
\chapter{Introdução}

% as epígrafes nos capítulos são opcionais
%\epigrafecap{The reasonable man adapts himself to the world; the unreasonable one persists in trying to adapt the world to himself. %Therefore all progress depends on the unreasonable man.}{George Bernard Shaw}

A computação em nuvem trouxe um novo modelo de relacionamento entre as empresas e os seus recursos computacionais. A gestão destes recursos sempre foi uma questão bastante delicada, que tem um reflexo direto no gerenciamento dos custos das companhias. E embora haja organizações que têm uma demanda por recursos computacionais com um comportamento mais previsível, é bastante significativo o número de empresas que têm necessidades mais dinâmicas neste aspecto.

Um exemplo clássico de comportamento com alto grau de variação é encontrado no comércio eletrônico. A sazonalidade com que os recursos são demandados faz com que as empresas gastem muito dinheiro em infraestrutura para garantir o desempenho desejado durante os picos de acesso, e fiquem com recursos parados durante a baixa temporada. É claro, que este cenário não é único do comércio eletrônico, na verdade a maioria das empresas, em maior ou menor grau, acabam apresentando um comportamento semelhante a este no que diz respeito às suas necessidades computacionais.

A solução oferecida pela cloud computing é justamente disponibilizar recursos computacionais para as empresas de acordo com suas respectivas demandas. Desta forma, os usuários não precisam arcar com o alto custo de manter um parque de máquinas, ou mesmo com as questões relacionadas a gestão e aquisição delas. E tudo isto a um custo bastante baixo, com uma disponibilidade imediata de contratação e um gerenciamento bastante simplificado.

Neste contexto, uma das características mais importantes da computação em nuvem é a elasticidade. Ou seja, a capacidade do ambiente computacional aumentar ou diminuir os recursos demandados pelo usuário. Esta capacidade se apresenta tanto em nível de instâncias de máquinas virtuais e/ou nós de computação, quanto em nível de CPU, memória e largura de banda de rede. Uma lista mais completa das características essenciais de um ambiente em nuvem pode ser encontrada em \cite{coutinho}, que cita \textit{"elasticidade rápida"} como uma destas características.

A política de alocação de recursos em ambientes de computação em nuvem pode ser tanto manual quanto automática. Para a abordagem automática, temos ainda duas classificações: reativa e proativa. Na primeira, o usuário define marcos (thresholds) para um mecanismo de regra-condição-ação através dos quais o sistema aloca mais recursos (threshold superior) ou os consolida (threshold inferior). Embora, esta primeira abordagem seja chamada de automática, se faz necessária a intervenção do usuário para especificar as regras, bem como conhecimento da ferramenta de nuvem. Por outro lado, a abordagem proativa usa técnicas de predição para identificar padrões e antecipar a alocação ou consolidação de recursos de forma automática, sem a intervenção do usuário.

A abordagem reativa, juntamente com as políticas manuais, é mais comum em sistemas comerciais de nuvem disponíveis na web. Enquanto que existe um esforço de pesquisas na academia para propor soluções proativas para a questão da elasticidade. Portanto, um dos grandes desafios para os provedores é justamente retirar a necessidade de intervenção do usário e oferecer mecanismos que implementem a elasticidade proativa. O que traria um ajuste mais fino na economia de recursos, dentre outros benefícios.

\section{Questão de Pesquisa}
Um dos grandes pontos a serem discutidos, no que diz respeito à implementação de um tratamento automático para a elasticidade, é a questão da análise do comportamento da aplicação. A qual oferece seus próprios obstáculos a serem transpostos. Uma técnica de auto-escalonamento que tenta antecipar os requisitos futuros de uma aplicação, baseado na análise de seu desempenho, é portanto, uma abordagem preditiva. Na literatura, este tema tem sido discutido sob diversos pontos de vista, como nos mostra o trabalho de \cite{Lorido-botr2012}. Algumas das técnicas sugeridas apresentam tanto uma natureza reativa quanto proativa, enquanto outras são excencialmente proativas. A análise preditiva baseada em séries temporais é identificada por \cite{Lorido-botr2012} como uma destas estratégias que têm uma abordagem puramente proativa. Uma série temporal é uma coleção de observações, as quais podem ser discretizadas ou não, feitas sequencialmente ao longo de um período de tempo. Ou como definido por \cite{Shumway2000}, séries temporais podem ser definidas como uma coleção de variáveis aleatórias indexadas de acordo com a ordem em que elas são obtidas no tempo. 

Posto isto, nossa questão de pesquisa é:
\begin{itemize}
	\item A análise de séries temporais pode ser usada para encontrar padrões na carga de trabalho de uma aplicação, com o intuito de prever valores futuros, de forma que ofereça subsídios para a implementação de uma solução de elasticidade proativa?
\end{itemize}

\section{Objetivos}
O objetivo deste trabalho é:

\begin{itemize}
	\item Apresentar uma proposta para análise preditiva que possa servir de base na implementação da elasticidade proativa em ambientes de computação em nuvem.
\end{itemize}

Para atingir o objetivo acima citado, temos os seguintes objetivos secundários:
\begin{itemize}
	\item Desenvolver uma estratégia para coleta de dados;
	\item Criar uma estrutura para analisar os dados e aplicar modelos matemáticos condizentes com os padrões encontrados;
	\item Disponibilizar como resposta uma sugestão de alteração da configuração vigente, em virtude da demandada prevista de acordo com a análise realizada. 
\end{itemize}


\section{Organização do Texto}
Este texto está organizado em cinco seções. Primeiramente, na seção 2, iremos revisar questões básicas a respeito da área da computação em nuvem, e as definições que são pertinentes ao escopo deste trabalho. Uma retomada dos tópicos relacionados às séries temporais também se faz necessária, bem como questões gerais a respeito de análise preditiva. Na seção 3, iremos avaliar trabalhos correlatos na área, e suas contribuições para a questão da elasticidade proativa. Esta avaliação tomará como objetivo traçar um quadro comparativo das soluções propostas, suas divergências e convergências, com o intuito de identificar lacunas e oportunidades de trabalho na literatura analisada. A seção 4 trará o modelo proposto, descrevendo sua arquitetura e metodologia para avaliação. E por fim, a seção 5 apresenta uma conclusão, especificando as contribuições esperadas, apontando trabalhos relacionados e um cronograma para o projeto.

%=======================================================================
% Escrevendo o Texto
%=======================================================================

%
%	FUNDAMENTAÇAO TEORICA
%
\chapter{Fundamentação Teórica}

\section{Computação em Nuvem}
Segundo \cite{Lorido-botr2012}, cloud computing é uma tecnologia emergente que está se tornando muito popular. E grande parte disso devesse a sua característica elástica, a qual permite aos usuários que adquiram e liberem recursos sob demanda, e paguem somente pelos recursos que eles precisam (modelo pay-as-you-go). Estes recursos se apresentam geralmente na forma de Máquinas Virtuais (Virtual Machines - VM). Há três principais modelos de serviços associados com a computação em nuvem, são eles:

%TODO conseguir referências bibliográficas para os exemplos citados abaixo
\begin{itemize}
	\item Infraestrutura como um serviço (IaaS) designa o provisionamento de recursos como rede, processamento, armazenamento e banda. Normalmente o usuário não administra ou controla a infraestrutura da nuvem, mas tem gerenciamento sobre os sistemas operacionais, armazenamento, aplicativos implantados, etc . Exemplos de infraestrutura como serviço são: Amazon EC2 e Google Compute Engine.
	\item Platform-as-a-Service (PaaS) designa ambientes de programação e ferramentas suportadas pelos provedores de nuvem que podem ser usadas pelos consumidores para construir e implantar aplicações na infraestrutura da nuvem. Alguns exemplos de plataforma como serviços são: Amazon Elastic Beanstalk, Heroku, Google App Engine e Microsoft Windows Azure.
	\item Software-as-a-Service (SaaS) designa aplicações hospedadas pelo fornecedor, sobre as quais o cliente não tem visibilidade da plataforma ou infraestrutura em que está implantada. Exemplos: Google Apps e Microsoft Office 365.
\end{itemize}

A computação em nuvem também pode ser caracterizada pelo seu modelo de implantação, o qual define restrição ou abertura de acesso, dentre outras coisas. Tais modelos podem ser: nuvem privada, nuvem pública, nuvem comunidade e nuvem híbrida. Para os objetivos deste trabalho, esta divisão é indiferente. Uma melhor definição destes modelos pode ser encontrada em \citep{coutinho}.

\section{Elasticidade}

Sem dúvida uma das principais razões para que a computação em nuvem seja utilizada é a capacidade de adquirir recursos de uma maneira dinâmica. Esta característica faz com que, para o cliente, a nuvem pareça possuir recursos computacionais infinitos. Por outro lado, para o provedor, o desafio de manter esta ilusão é enorme, pois é muito difícil compreender quais são os requisitos de elasticidade de uma aplicação e da carga de trabalho que ela apresenta.

Segundo o National Institute of Standards and Technology (NIST) em sua definição de cloud computing \cite{Mell2012}, o conceito de elasticidade é a capacidade de rápido provisionamento e desprovisionamento, com capacidade de recursos virtuais praticamente infinita e quantidade adquirível sem restrição a qualquer momento. \citep{Righi2013} também cita a importância da noção de tempo para o conceito de elasticidade e explica a diferença entre esta e a escalabilidade, conceitos muitas vezes confundidos. A escalabilidade se relaciona à capacidade do sistema ser expandido e este conceito é livre da noção de tempo.
%TODO - FIX referência a Coutinho não está sendo exibida
Existem várias definições para elasticidade, e em \cite{Coutinho} encontramos uma análise das diferentes definições, após o que o autor sugere sua própria definição para o conceito: \textit{"Capacidade de adicionar e remover recursos de forma automática de acordo com a carga de trabalho sem interrupções e utilizando os recursos de forma otimizada"}.

Nos próximos parágrafos abordaremos algumas características da elasticidade.

\subsection{Ação de alocação de recursos}
A ação de alocação de recursos, ou modalidade da ação da elasticidade pode ser categorizada de duas formas: 

\begin{enumerate}
	\item Escalonamento horizontal, segundo \cite{Lorido-botr2012} é adicionar réplicas de novos servidores e balanceadores de carga para distribuir a carga entre todas as réplicas disponíveis. Ou seja, é possível aumentar ou diminuir o número de instâncias de máquinas virtuais, ou até mesmo migrá-las para novos nós de processamento.
	\item Escalonamento vertical é a mudança de alocação de recursos em tempo de execução para uma máquina que está em operação, com alocação ou desalocação de memória, CPU, rede, etc.
\end{enumerate}

\subsection{Política de alocação de recursos}
Quanto às políticas que regem o gerenciamento de recursos pela escalabilidade, temos duas abordagens segundo \cite{Galante2012}:

\begin{enumerate}
	\item Manual quando há necessidade de uma ação do usuário para alterar o aprovisionamento de recursos, a qual pode se dar através de uma interface de programação (API) ou uma interface de gerenciamento oferecida pelo \textit{middleware}.
	\item Automática quando o provedor inclui um serviço completo de monitoramento e alocação automática de recursos. Esta abordagem ainda se subdivide em reativa e proativa.
	A política automática reativa utiliza-se de mecanismos do tipo regra-condição-ação, estabelecendo um limite superior, que quando atingido, dispara regras específicas que fazem a ação de alocar recursos de acordo com o que está estipulado nas referidas regras. De igual forma, um limiar inferior serve para consolidar recursos quando atingido o limite estipulado. Estes limiares são conhecidos na literatura como \textit{thresholds}.
\end{enumerate}	
	Por outro lado, a política proativa usa técnicas de predição para automaticamente escalar recursos. O objetivo final de um sistema de auto escalonamento de recursos é ajustar a alocação e consolidação de recursos automaticamente para minimizar custos enquanto garante o cumprimento dos objetivos de nível de serviço (SLO), conforme \cite{Lorido-botr2012}.
	A questão do escalonamento automático pode ser implementada por diversas abordagens e uma discussão bastante extensa a este respeito pode ser encontrada em \cite{Lorido-botr2012}, que faz uma classificação de técnicas puramente reativas, proativas e cita algumas mistas. 	Para o escopo deste trabalho focaremos na abordagem automática proativa, com o uso de séries temporais, as quais serão discutidas nas próximas seções, as demais abordagens não serão discutidas neste trabalho, salvo casos em que se faça necessário abordar alguma outra estratégia em função de algum trabalho relacionado - citado na seção 3 - que a utilize.

\subsection{Implementação de elasticidade}
Os métodos empregados na implementação das soluções de elasticidade encontrados na literatura podem ser categorizados em três tipos: replicação, migração e redimencionamento. A replicação nada mais é que o escalonamento horizontal, descrito anteriormente. E é sem dúvida o método mais utilizado para prover elasticidade, tanto em provedores comerciais quanto em trabalhos na academia \cite{Galante2012}. O redimencionamento é a implementação do escalonamento vertical, o trabalho de \cite{Wilkes2010} emprega esta técnica e iremos analisá-lo melhor na seção de trabalhos relacionados. Por último, a migração é a transferência de uma máquina virtual entre servidores, geralmente esta técnica é usada para simular o comportamento obtido com o redimensionamento em nuvens que não permitem este tipo de ação \cite{Galante2012}.

\section{Análise de Séries Temporais}
Para analisar séries temporais, temos dois caminhos que podem ser usados individualmente ou em conjunto. Um é a abordagem do domínio do tempo e a outra a abordagem do domínio de frequência. No domínio do tempo, focamos em analisar a correlação entre pontos adjacentes de uma série com a dependência dos valores atuais com os valores no passado. Já na abordagem de domínio de frequência, nos preocupamos com as características de periodicidade. Na maioria dos casos o resultado obtido da aplicação dessas abordagens é semelhante para séries longas, mas segundo \cite{Shumway2000}, o desempenho para observações curtas é melhor quando realizado no domínio do tempo.
Vamos primeiramente entender melhor o que são as séries temporais e suas características mais fundamentais antes de nos aprofundarmos na questão da análise propriamente dita.

\section{Séries Temporais}
De acordo com \cite{Ehlers2009}, uma série temporal é uma coleção de observações feitas sequencialmente ao longo do tempo. Ele aponta como uma das características mais importantes deste tipo de dado o fato de haver dependência entre observações vizinhas. Ou seja, em séries temporais a ordem dos dados tem grande relevância. As séries temporais estão presentes em vários campos do conhecimento tais como: economia, medicina e metereologia. Elas podem ser utilizadas para encontrar padrões em uma carga de trabalho de um servidor, por exemplo. Ou para tentar predizer os valores futuros desta carga. Em ambos os casos a informação necessária para aplicação da técnica é uma lista das \textit{n} últimas observações da série temporal, chamada \textit{janela de entrada}, ou matematicamente conhecido como \textit{espaço de estados}.

\subsection{Conceitos básicos}
Quando realizamos observações continuamente no tempo, chamamos de série contínua. As séries também podem ser ditas discretas, quando as observações são realizadas em tempos específicos, normalmente com períodos iguais. Também podemos discretizar séries contínuas ao registrar observações em determinados intervalos de tempo.

\cite{Ehlers2009} cita os seguintes objetivos gerais para estudo de séries temporais:

\begin{itemize}
	\item Descrever propriedades da série, seus padrões, existência de sazonalidade ou variação cíclica.
	\item Explicar variações em um determinada série através das variações observadas em outra série temporal.
	\item Controlar processos através da análise de séries temporais, através das quais podemos identificar fatores de qualidade.
	\item Predizer valores futuros com base em observações passadas. 
\end{itemize}

Este último objetivo das séries temporais é objeto de estudo deste trabalho.

\subsubsection{Sazonalidade, tendência e variação cíclica}

Quando uma série apresenta um comportamento que se repete ao longo do tempo, em intervalos específicos normalmente anuais, dizemos que esta apresenta um comportamento sazonal. Para ilustrar: é natural imaginarmos um aumento no número de vendas de uma determinada loja de brinquedos no período próximo ao natal, ou o aumento de requisições de um determinado site de comércio eletrônico em datas comemorativas. 

Outra característica importante pertinente às séries temporais é a tendência. Esta pode ser de crescimento ou decrescimento, e apresenta diversos padrões: linear, exponencial e de amortecimento. No primeiro, observamos um padrão mais ou menos constante, no segundo a alteração dentro de um período de tempo está atrelada a um fator. Por exemplo, aumento de vendas em 1.4 ao ano. E por último, o padrão de amortecimento apresenta uma variação percentual em relação ao aumento do período anterior. Um aumento de 50 \% em vendas em relação ao ano anterior, se no primeiro ano tivermos 1000 vendas esperadas, então no segundo teremos 500 e assim por diante.

%TODO citar variação cíclica

\subsubsection{Correlação}

%TODO explicar este conceito que é o mais importante em séries


\subsection{Processos Estocásticos}
Processos estocásticos são modelos probabilísticos adequados ao tratamento de dados de séries temporais. Podemos definir matematicamente processos estocásticos, segundo \cite{Ehlers2009}, da seguinte forma: uma coleção de variáveis aleatórias ordenadas no tempo e definidas em um conjunto de pontos \textit{T}, podendo ser contínuo ou discreto como mencionado anteriormente neste trabalho. A variável aleatória no tempo \textit{t}, pode ser denotada por \textit{X(t)} quando estivermos tratando do caso contínuo, no qual normalmente temos:

\[-\infty < \textit{t} < \infty\]

para o caso discreto, com

\[\textit{t} = 0,\pm1,\pm2...,\]

denotaremos a variável aleatória por \textit{X}_{t}.\newline

Um processo estocástico pode ser definido portanto da seguinte maneira: 

\[\textit{X(t}_{1}),..., \textit{X(t}_{k})\]

para qualquer conjunto de tempos e valor de } \textit{k}.\newline

Para descrever um processo estocástico, normalmente utilizamo-nos da média, variância e autocovariância. A média é o valor que aponta onde mais se concentram os dados da distribuição. Funciona como um ponto de equilíbrio das frequências. A variância de uma variável aleatória denota a medida de sua dispersão. Ou seja, demonstra quão longe os seus valores estão de um dado valor esperado. E por sua vez a autocovariânica é uma medida do grau de interdependência entre duas variáveis aleatórias. É também chamada de dependência linear.Estas funções são definidas abaixo respectivamente:

\[ \mu \textit{(t)} = E[\textit{X(t)}] \]
\[ \sigma^{2} \textit{(t)} = Var[\textit{X(t)}] \]
\[\gamma \textit{(t}_{1} \textit{,t}_{2}) = E[ \textit{X(t}_{1}) - \mu \textit{(t}_{1})] [ \textit{X(t}_{2}) - \mu \textit{(t}_{2})] \]

Existem muitos processos estocásticos que são utilizados na especificação de modelos para séries temporais. Um levantamento de trabalhos que utilizam séries temporais na predição de valores futuros para cargas de trabalhos na construção de uma abordagem proativa de elasticidade foi realizado por \citep{Lorido-botr2012}. Neste trabalho ele cita as seguintes processos estocásticos: processos de médias móveis, processos autoregressivos, modelos mistos ARMA, suavização exponencial e abordagens baseadas em aprendizado de máquina.\citep{Lorido-botr2012} classifica estas técnicas da seguinte maneira:

\begin{itemize}
	\item Métodos de médias - podem ser usados para suavizar a série temporal com o objetivo de remover ruído ou para fazer predições. A suavização visa remover discrepâncias (\textit{outliers}, as quais podem ter um efeito devastador no processo de predição. Nesta categoria encontramos as médias móveis e a suavização exponencial, por exemplo.
	\item Métodos autorregressivos - utilizados para predições. %TODO adicionar exemplos
	\item Processos autoregressivos médias móveis (ARMA) - combina ambos os métodos anteriores.
\end{itemize}
Uma descrição bastante detalhada destes métodos pode ser encontrada em \cite{Lee2009} e em \cite{Shumway2000}.

%=======================================================================
% TRABALHOS RELACIONADOS
%=======================================================================
\chapter{Trabalhos Relacionados}


%=======================================================================
% MODELO PROPOSTO
%=======================================================================
\chapter{Modelo}

%=======================================================================
% CONCLUSÃO
%=======================================================================
\chapter{Conclusão}

%=======================================================================
% Referências
%=======================================================================
\bibliography{tcc1}


%=======================================================================
% Exemplo de Apêndice
% O Apêndice é utilizado para apresentar material complementar elaborado
% pelo próprio autor.  Deve seguir as mesmas regras de formatação do
% corpo principal do documento.
%=======================================================================
\appendix
\chapter{Informações Complementares}

O Apêndice é o lugar para incluir textos complementares, que não são essenciais para o entendimento do assunto principal da monografia, mas que podem contribuir com informação relevante (por exemplo, uma prova matemática, uma conceituação básica, etc.).  Ele deve seguir o formato normal do documento.

%=======================================================================
% Exemplo de Anexo
% O Anexo é utilizado para a ``inclusão de materiais não elaborados pelo
% próprio autor, como cópias de artigos, manuais, folders, balancetes, etc.
% e não precisam estar em conformidade com o modelo''.
%=======================================================================
\annex
\chapter{Artigos Publicados}
Existe diferença entre os Apêndices e os Anexos.  Os apêndices trazem informação escrita pelo próprio autor do trabalho, incorporando-se ao formato da monografia como um todo.  Já um anexo é um material à parte, definido/publicado por si só, e que o autor julga conveniente ser apresentado juntamente com a monografia.  Normalmente também vai apresentar formato próprio, como um artigo publicado, um folder, uma planilha, etc.
\end{document}
